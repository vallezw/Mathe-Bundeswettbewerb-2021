% If you're new to LaTeX, here's some short tutorials:
% https://www.overleaf.com/learn/latex/Learn_LaTeX_in_30_minutes
% https://en.wikibooks.org/wiki/LaTeX/Basics
\documentclass{article}
% Formatting
%\usepackage[utf8]{inputenc}
%\usepackage[T2A]{fontenc}
\usepackage[margin=1.9685in]{geometry}
\usepackage[titletoc,title]{appendix}
%\usepackage[russian]{babel}
\usepackage{amssymb}
% Images
% https://www.overleaf.com/learn/latex/Inserting_Images
% https://en.wikibooks.org/wiki/LaTeX/Floats,_Figures_and_Captions
%\usepackage{float}
%\usepackage{graphicx}
% Title content
\title{Bundeswettbewerb Mathematik 2021}
\author{Valentin Zwerschke}
\date{16 Februar, 2021}

%\newfontfamily\russianfont[Script=Cyrillic]{TrueType}

\begin{document}
\maketitle

\section*{Aufgabe 1}
\section*{Aufgabe 2}
\section*{Aufgabe 2 a}
Den Bruch $\frac{3}{2021}$ als Stammbruch einer ersten Zahl $n$ und einem Rest.
\begin{center}
    $\frac{3}{2021} = \frac{1}{n} + (\frac{3}{2021} - \frac{1}{n})$
\end{center}
In einem nächsten Schritt machen wir die Brüche der Differenz in der Klammer gleichnahmig. Mit der Primzahlzerlegung $2021 = 43 \cdot 47$ unterscheiden wir dabei 4 Fälle.
\begin{itemize}
    \item $n = 43 \cdot m$
    \item $n = 47 \cdot m$
    \item $n = 43 \cdot 47 \cdot m$
    \item $n = m$
\end{itemize}
Wobei \(m \in \mathbb{N}\) die nicht durch die Primzahlen (43 oder 47) teilbar ist.\\Im ersten Fall lautet der Zweite Summand im Term \(\frac{3m-47}{43 \cdot 47m}\). Damit dies ein Stammbruch wird müsste \(3m = 48\) sein. Dies führt zu keiner Lösung, da \(m\) ganzzahlig sein muss.\\Im zweiten Fall müsste \(3m = 44\) sein, was ebenfalls keine Lösung bringt.\\
Im drittem Fall finden wir mit der gleichen Methode: \(3m = 2\). Auch dies führt zu keiner Lösungen.\\
Im viertem Fall müsste \(3m = 47 \cdot 43 + 1 = 2022\) sein. Mit \(m = 674\) ist hier eine Lösung gefunden. Die zugehörige Stammbruchzerlegung lautet \(\frac{3}{2021} = \frac{1}{674} + \frac{1}{43 \cdot 47 \cdot 674}\).\\
Damit ist bewiesen, dass es nur genau eine Zerlegung ein zweier Stammbrüche gibt bei \(\frac{3}{2021}\).
\section*{Aufgabe 2 b}
In der zweiten Aufgabe ist eine nicht durch 3 teilbare Zahl \(n\) gesucht, 
sodass \(\frac{3}{n}\) in 2021 verschiedene paare zweier Stammbrüche zerlegen werden kann.\\
Mit dem gleichen Ansatz aus Aufgabenteil A schreiben wir \(\frac{3}{n} = \frac{1}{m} + (\frac{3}{n} - \frac{1}{m})\). 
Wobei \(m\) eine beliebige Natürlich Zahl ist. Wir müssen also nun wieder prüfen bei welcher Wahl von \(m\), 
\(\frac{3}{n} = \frac{1}{m}\) ein Stammbruch ist. Meine Hyptothese lautet, dass \(n = 2^{2 \cdot l}\) mit \(l \in \mathbb{N}\) 
gute Chancen auf eine Lösung hat. Wie im Aufgabenteil A müssen wir die beiden Brüche gleichnamig machen.\\
Hierfür benutzten wie wieder die Fallunterscheidung für $m$. \(m = m_1, m_2, m_3, ...\) mit \(m_i = 2^{i \cdot k}\). Jedes $m$ mit geradem Index führt zu keiner Lösung. Jedes $m$ mit ungeradem Index führt zu einer Lösung.\\
Für den Fall eines geradem Indexes ($m = m_{2j}$) gilt:
\begin{equation}
    \frac{3}{n} - \frac{1}{n} = \frac{3}{2^2l} 
\end{equation}
\section*{Aufgabe 3}
\section*{Aufgabe 4}

\end{document}
