% If you're new to LaTeX, here's some short tutorials:
% https://www.overleaf.com/learn/latex/Learn_LaTeX_in_30_minutes
% https://en.wikibooks.org/wiki/LaTeX/Basics
\documentclass{article}
% Formatting
%\usepackage[utf8]{inputenc}
%\usepackage[T2A]{fontenc}
\usepackage[margin=1.9685in]{geometry}
\usepackage[titletoc,title]{appendix}
%\usepackage[russian]{babel}
\usepackage{amssymb}
% Images
% https://www.overleaf.com/learn/latex/Inserting_Images
% https://en.wikibooks.org/wiki/LaTeX/Floats,_Figures_and_Captions
%\usepackage{float}
%\usepackage{graphicx}
% Title content
\title{Bundeswettbewerb Mathematik 2021}
\author{Valentin Zwerschke}
\date{16 Februar, 2021}

%\newfontfamily\russianfont[Script=Cyrillic]{TrueType}

\begin{document}
\maketitle

\section*{Aufgabe 1}

\section*{Aufgabe 2}
\section*{Aufgabe 2 a}
Den Bruch $\frac{3}{2021}$ als Stammbruch einer ersten Zahl $n$ und einem Rest.
\begin{center}
    $\frac{3}{2021} = \frac{1}{n} + (\frac{3}{2021} - \frac{1}{n})$
\end{center}
In einem nächsten Schritt machen wir die Brüche der Differenz in der Klammer gleichnahmig. Mit der Primzahlzerlegung $2021 = 43 \cdot 47$ unterscheiden wir dabei 4 Fälle.
\begin{itemize}
    \item $n = 43 \cdot m$
    \item $n = 47 \cdot m$
    \item $n = 43 \cdot 47 \cdot m$
    \item $n = m$
\end{itemize}
Wobei \(m \in \mathbb{N}\) die nicht durch die Primzahlen (43 oder 47) teilbar ist.\\Im ersten Fall lautet der Zweite Summand im Term \(\frac{3m-47}{43 \cdot 47m}\). Damit dies ein Stammbruch wird müsste \(3m = 48\) sein. Dies führt zu keiner Lösung, da \(m\) ganzzahlig sein muss.\\Im zweiten Fall müsste \(3m = 44\) sein, was ebenfalls keine Lösung bringt.\\
Im drittem Fall finden wir mit der gleichen Methode: \(3m = 2\). Auch dies führt zu keiner Lösungen.\\
Im viertem Fall müsste \(3m = 47 \cdot 43 + 1 = 2022\) sein. Mit \(m = 674\) ist hier eine Lösung gefunden. Die zugehörige Stammbruchzerlegung lautet \(\frac{3}{2021} = \frac{1}{674} + \frac{1}{43 \cdot 47 \cdot 674}\).\\
Damit ist bewiesen, dass es nur genau eine Zerlegung ein zweier Stammbrüche gibt bei \(\frac{3}{2021}\).
\section*{Aufgabe 2 b}
In der zweiten Aufgabe ist eine nicht durch 3 teilbare Zahl \(n\) gesucht, 
sodass \(\frac{3}{n}\) in 2021 verschiedene paare zweier Stammbrüche zerlegen werden kann.\\
Mit dem gleichen Ansatz aus Aufgabenteil A schreiben wir \(\frac{3}{n} = \frac{1}{m} + (\frac{3}{n} - \frac{1}{m})\). 
Wobei \(m\) eine beliebige natürliche Zahl ist. Wir müssen also nun wieder prüfen, bei welcher Wahl von \(m\), 
der Term \(\frac{3}{n} - \frac{1}{m}\) ein Stammbruch ist. Meine Hyptothese lautet, dass \(n = 2^{2 \cdot l}\) mit \(l \in \mathbb{N}\) 
gute Chancen auf eine Lösung hat. Wie im Aufgabenteil a müssen wir die beiden Brüche des Termes gleichnamig machen.\\
Hierfür benutzten wie wieder die Fallunterscheidung für $m$.\\
\(m = m_1, m_2, m_3, ..., m_{2l}\) mit \(m_i = 2^{i \cdot k}\). 
Die Differenz, die eine Stammbruch sein Soll lässt sich wie folgt umformen für $m_i$. 
\begin{equation}
    \frac{3}{n} - \frac{1}{m} = \frac{3}{2^{2l}} - \frac{1}{2i \cdot k} = \frac{3k - 2^{2l - i}}{2^{2l} \cdot k}
\end{equation}
Dies ist genau dann ein Stammbruch, wenn ein $k$ gefunden wurde, welches für dieses Term gilt: $3k - 2^{2l - i} = 1$.\\
Dies ist dann der Fall, wenn $i$ ungerade ist. Siehe Beweis weiter unten.\\
Wir haben also bei der Fallunterscheidung je eine Lösung für \(m_1, m_3, m_5, ..., m_{2l - 1}\).\\
Um genau 2021 Lösungen zu bekommen, setzten wir  $l = 2021$.\\\\
Nun kommen wir zu dem oben angekündigtem Beweis, dass \(2^{2n - 1} + 1\) durch 3 teilbar ist. Hierzu verwenden wir vollständige Induktion.\\
Es gelte für ein \(n \in \mathbb{N}\), dass der Term \(2^{2n - 1} + 1\) ist durch 3 teilbar ist. Für $n = 1$ ist dieses Trivial erfüllt. Für $n + 1$ gilt: 
\begin{equation}
    2^{2(n + 1) - 1} + 1 = 4 \cdot (2^{2n - 1} + 1) - 3
\end{equation}
Da beide Summanden durch 3 teilbar sind, ist der ganze Term durch 3 teilbar, was zu beweisen war.\\
Analog müssen wir noch beweisen, dass \(2^{2n} + 1\) nicht durch 3 teilbar ist. Hierzu verwenden wir wieder vollständige Induktion.\\
Die Behauptung gelte für ein \(n \in \mathbb{N}\). Für $n = 1$ ist der Term 5, was nicht durch 3 teilbar ist. Für den Schritt nach $n + 1$ gilt: 
\begin{equation}
    2^{2(n + 1)} + 1 = 4 \cdot (2^{2n} + 1) - 3
\end{equation}
Da der erste Summand nicht durch 3 teilbar ist, ist der ganze Term nicht mehr durch 3 teilbar.
\section*{Aufgabe 3}
\section*{Aufgabe 4}

\end{document}
